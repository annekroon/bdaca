% !TeX document-id = {f19fb972-db1f-447e-9d78-531139c30778}
% !BIB program = biber
\documentclass{beamer}
%\documentclass[handout]{beamer}

%\usetheme{focus}
%\usepackage[style=authoryear]{biblatex}
%\renewcommand*{\nameyeardelim}{\addcomma\addspace}
\usepackage[T1]{fontenc}
\usetheme[block=fill,subsectionpage=progressbar,sectionpage=progressbar]{metropolis} 
%\usepackage[natbibapa]{apacite}
\usepackage{wasysym}
\usepackage{etoolbox}
\usepackage[utf8]{inputenc}

\usepackage{threeparttable}
\usepackage{subcaption}

\usepackage{tikz-qtree}
\setbeamercovered{still covered={\opaqueness<1->{5}},again covered={\opaqueness<1->{100}}}


\usepackage{listings}

\lstset{
	basicstyle=\scriptsize\ttfamily,
	columns=flexible,
	breaklines=true,
	numbers=left,
	%stepsize=1,
	numberstyle=\tiny,
	backgroundcolor=\color[rgb]{0.85,0.90,1}
}

 
\usepackage[american]{babel}
\usepackage{csquotes}
\usepackage[style=apa, backend = biber]{biblatex}
\DeclareLanguageMapping{american}{american-UoN}
\addbibresource{../../bdaca.bib}
\renewcommand*{\bibfont}{\tiny}

\usepackage{tikz}
\usetikzlibrary{shapes,arrows,matrix}
\usepackage{multicol}

 
\makeatletter
\setbeamertemplate{headline}{%
  \begin{beamercolorbox}[colsep=1.5pt]{upper separation line head}
  \end{beamercolorbox}
  \begin{beamercolorbox}{section in head/foot}
    \vskip2pt\insertnavigation{\paperwidth}\vskip2pt
  \end{beamercolorbox}%
  \begin{beamercolorbox}[colsep=1.5pt]{lower separation line head}
  \end{beamercolorbox}
}
\makeatother

 

\setbeamercolor{section in head/foot}{fg=normal text.bg, bg=structure.fg}



\newcommand{\question}[1]{
	\begin{frame}[plain]
		\begin{columns}
			\column{.3\textwidth}
			\makebox[\columnwidth]{
				\includegraphics[width=\columnwidth,height=\paperheight,keepaspectratio]{../../pictures/mannetje.png}}
			\column{.7\textwidth}
			\large
			\textcolor{orange}{\textbf{\emph{#1}}}
		\end{columns}
\end{frame}}




\title[Big Data and Automated Content Analysis]{\textbf{Big Data \& Automated Content Analysis} \\ Week 2 -- Wednesday: »Programming concepts for data analysis«}
\author[Damian Trilling]{Damian Trilling \\ ~ \\ \footnotesize{d.c.trilling@uva.nl \\@damian0604} \\ \url{www.damiantrilling.net}}
\date{10 February 2021}
\institute[UvA]{Afdeling Communicatiewetenschap \\Universiteit van Amsterdam}

\begin{document}

\begin{frame}{}
\titlepage
\end{frame}

\begin{frame}{Today}
\tableofcontents
\end{frame}







\section{Datatypes}


\begin{frame}{Python lingo}
	\begin{block}{Basic datatypes (variables)}
		\begin{description}
			\item[{\color{red}int}] \texttt{37}
			\item[{\color{red}float}] \texttt{1.75}
			\item[{\color{red}bool}] \texttt{True}, \texttt{False}
			\item[{\color{red}string}] \texttt{"Alice"}
			\onslide<2->{\scriptsize \item[({\color{red}variable name}] \texttt{firstname})}
		\end{description}
	\end{block}
	\onslide<2->{\textbf{"firstname" and firstname is not the same.\\}}
	\onslide<3->{\textbf{"5" and 5 is not the same.}\\ But you can transform it: {\tt{int("5")}} will return 5.}\\
	\onslide<3->{\textbf{You cannot calculate \texttt{3 * "5"}} {\tiny{(In fact, you can. It's \tt{"555"})}}.\\
		But you can calculate {\tt{3 * int("5")}}}
\end{frame}



\begin{frame}{Python lingo}
	\begin{block}{More advanced datatypes}
		\begin{description}
			\item[{\color{red}list}]<2-> \texttt{firstnames = ['Alice','Bob','Cecile'] \\
				 lastnames = ['Garcia','Lee','Miller']}
			\item[{\color{red}list}]<3->\texttt{ages = [18,22,45]}
			\item[{\color{red}dict}]<4-> \texttt{agedict = \{'Alice': 18, 'Bob': 22, 'Cecile': 45\} }
		\end{description}
		\onslide<5->{
		Note that the elements of a list, the keys of a dict, and the values of a dict can have any* datatype! (You can even mix them, but it's better to be consistent!)
		
		\tiny{*Well, keys cannot be mutable $\rightarrow$ see book}}
	\end{block}
\end{frame}


\begin{frame}{Python lingo}
	\begin{block}{Retrieving specific items}
		\begin{description}
			\item[{\color{red}list}]<1-> \texttt{firstnames[0]} gives you the first entry\\ 
				\texttt{firstnames[-2]} gives you the one-but-last entry\\
				\texttt{firstnames[:2]} gives you entries 0 and 1\\
				\texttt{firstnames[1:3]} gives you entries 1 and 2\\
				\texttt{firstnames[1:]} gives you entries 1 until the end\\
			\item[{\color{red}dict}]<2-> \texttt{agedict["Alice"]} gives you 18 
		\end{description}
	
	\end{block}
\end{frame}


\question{Think of at least two different ways of storing data about some fictious persons (first name, last name, age, phone number, \ldots) using lists and/or dictionaries. What are the pros and cons?}


\begin{frame}{Python lingo}
	\begin{block}{Less frequent, but still useful datatypes}
		\begin{description}
			\item[{\color{red}set}]A collection in which each item is unique: \texttt{\{1,2,3\}}
			\item[{\color{red}tuple}]Like a list, but \emph{immutable}: \texttt{(1,2,2,2,3)}
			\item[{\color{red}defaultdict}]A dict that does not raise an error but returns the ``empty'' value of its datatype (0 for int, "" for str) if you try access a non-existing key (great for storing results and counting things!)
			\item[{\color{red}np.array}]A list-like datatype provided by the \texttt{numpy} package optimized for efficient mathematical operations.
			\item[$\ldots$]\ldots
		\end{description}
	\end{block}
\small{You will come across more later}
\end{frame}



\section{Functions and methods}
\begin{frame}{Python lingo}
	\begin{block}{Functions}
		\begin{description}
			\item[{\color{red}functions}]<2-> Take an input and return something else \\ {\tt{int(32.43})} returns the integer 32. \texttt{len("Hello")} returns the integer 5.\\ 
			\item[{\color{red}methods}]<3-> are similar to functions, but directly associated with an object. {\tt{"SCREAM".lower()}} returns the string "scream"
		\end{description}
	\end{block}
	\onslide<4->{Both functions and methods end with \texttt{()}. Between the \texttt{()}, \emph{arguments} can (sometimes have to) be supplied.}
\end{frame}



\begin{frame}[fragile]{Some functions}
\begin{lstlisting}
len(x)        # returns the length of x
y = len(x)    # assign the value returned by len(x) to y
print(len(x)) # print the value returned by len(x)
print(y)      # print y
int(x)        # convert x to an integer
str(x)        # convert x to a string
sum(x)        # get the sum of x
\end{lstlisting}
\end{frame}

\question{How could you print the mean (average) of a list of integers using the functions on the previous slide?}



\begin{frame}[fragile]{Some methods}
Some string methods
\begin{lstlisting}
mystring = "Hi! How are you?"
mystring.lower()   # return lowercased string (doesn't change original!)
mylowercasedstring =  mystring.lower()  # save to a new variable
mystring =  mystring.lower()  # or override the old one
mystring.upper()   # uppercase
mystring.split()   # Splits on spaces and returns a list ['Hi!', 'How', 'are', 'you?']
\end{lstlisting}

We'll look into some list methods later.

\textbf{$\Rightarrow$ You can use TAB-completion in Jupyter to see all methods (and properties) of an object!}
\end{frame}


\begin{frame}[fragile]{Writing own functions}
You can write an own function:
\begin{lstlisting}
def addone(x):
    y = x + 1
    return y
\end{lstlisting}

Functions take some input (``argument'') (in this example, we called it \texttt{x}) and \emph{return} some result.

Thus, running
\begin{lstlisting}	
addone(5)
\end{lstlisting}
returns \tt{6}.
\end{frame}

\begin{frame}[fragile]{Writing own functions}

\begin{alertblock}{Attention, R users! (maybe obvious for others?)}
	You \emph{cannot}* apply the function that we just created on a whole list -- after all, it takes an int, not a list as input.
\end{alertblock}

(wait a sec foruntil we cover for loops later today, but this is how you'd do it (by calling the function for each element in the list separately):):

\begin{lstlisting}
mynumbers = [5, 3, 2, 4]
results = [addone(e) for e in mynumbers]
\end{lstlisting}

\vspace{1cm}
\tiny{* Technically speaking, you could do this by wrapping the \texttt{map} function around your own function, but that's not considered ``pythonic''. Don't do it ;-) \\}
\end{frame}




\section[Modifying lists \& dicts]{Modifying lists and dictionaries}
\begin{frame}[fragile]{Modifying lists}
	Let's use one of our first \textcolor{red}{method}s! Each \emph{list} has a method \texttt{.append()}:
\begin{block}{Appending to a list}
\begin{lstlisting}
mijnlijst = ["element 1", "element 2"]
anotherone = "element 3"   # note that this is a string, not a list!
mijnlijst.append(anotherone)
print(mijnlijst)
\end{lstlisting}
gives you:
\begin{lstlisting}
["element 1", "element 2", "element 3"]
\end{lstlisting}
\end{block}
\end{frame}



\begin{frame}[fragile]{Modifying lists}
\begin{block}{Merging two lists (= extending)}
\begin{lstlisting}
mijnlijst = ["element 1", "element 2"]
anotherone = ["element 3", "element 4"]
mijnlist.extend(anotherone)
print(mijnlijst)
\end{lstlisting}
gives you:
\begin{lstlisting}
["element 1", "element 2", "element 3", "element 4]
\end{lstlisting}
\end{block}
\end{frame}

\question{What would have happened if we had used \texttt{.append()} instead of \texttt{.extend()}?}

\question{Why do you think that the Python developers implemented \texttt{.append()} and \texttt{.extend()} as methods of a list and not as functions?}


\begin{frame}[fragile]{Modifying dicts}
\begin{block}{Adding a key to a dict (or changing the value of an existing key)}
\begin{lstlisting}
mydict = {"whatever": 42, "something": 11}
mydict["somethingelse"] = 76
print(mydict)
\end{lstlisting}
gives you:
\begin{lstlisting}
{'whatever': 42, 'somethingelse': 76, 'something': 11}
\end{lstlisting}
If a key already exists, its value is simply replaced.
\end{block}
\end{frame}




\section{for, if/elif/else, try/except}

\begin{frame}[fragile]{How can we structure our program?}
If we want to \emph{repeat} a block of code, exectute a block of code only \emph{under specific conditions}, or more generally want to structure our code, we use \emph{indention}.

	\begin{block}{Indention: The Python way of structuring your program}
		\begin{itemize}
			\item Your program is structured by TABs or SPACEs.
			\item Jupyter (or your IDE) handles (guesses) this for you, but make sure to not interfere and not to mix TABs or SPACEs!
			\item Default: four spaces per level of indention.
		\end{itemize}
	\end{block}
\end{frame}



\begin{frame}[fragile]{Indention}
\begin{block}{Structure}
	A first example of an indented block -- in this case, we want to \emph{repeat} this block:
\end{block}
\begin{lstlisting}
agedict = {'Zeus': None, 'Denis': 96, 'Alice': 18, 'Rebecca': 20 , 'Bob': 22, 'Cecile': 45}

myfriends = ['Alice','Bob','Cecile']

print ("The names and ages of my friends:")
for buddy in myfriends:
	print (f"My friend {buddy} is {agedict[buddy]} years old")
\end{lstlisting}

Output:
\begin{lstlisting}
My friend Alice is 18 years old
My friend Bob is 22 years old
My friend Cecile is 45 years old
\end{lstlisting}
\end{frame}

\begin{frame}[fragile]{What happened here?}

\begin{lstlisting}
for buddy in myfriends:
    print (f"My friend {buddy} is {agedict[buddy]} years old")
\end{lstlisting}

\small
	\begin{block}{The for loop}
\begin{enumerate}
	\item Take the first element from \texttt{myfriends} and call it \texttt{buddy} (like \texttt{buddy = myfriends[0]}) (line 1)
	\item Execute the indented block (line 2, but could be more lines)
	\item Go back to line 1, take next element  (like \texttt{buddy = myfriends[1]}) 
	\item Execture the indented block \ldots
	\item \ldots repeat until no elements are left \ldots
\end{enumerate}
	\end{block}
	
	\begin{block}{The f-string (\emph{formatted} string)}
If you prepend a string with an \texttt{f}, you can use curly brackets texttt{\{\}} to insert the value of a variable
	\end{block}
	
\end{frame}




\begin{frame}[fragile]{What happened here?}
\begin{lstlisting}
for buddy in myfriends:
    print (f"My friend {buddy} is {agedict[buddy]} years old")
\end{lstlisting}
\small
The line \emph{before} an indented block starts with a \emph{statement} indicating what should be done with the block and ends with a \texttt{:}

\footnotesize
\begin{block}{More in general, the \texttt{:} + indention indicates that}<2->
	\begin{itemize}
		\item<3-> the block is to be executed repeatedly (\texttt{for} statement) – e.g., for each element from a list, or until a condition is reached (\texttt{while} statement)
		\item<4-> the block is only to be executed under specific conditions (\texttt{if}, \texttt{elif}, and \texttt{else} statements)
		\item<5-> an alternative block should be executed if an error occurs in the block (\texttt{try} and \texttt{except} statements)
		\item<6-> a file is opened, but should be closed again after the block has been executed (\texttt{with} statement)
	\end{itemize}
\end{block}
\end{frame}



\begin{frame}[fragile]{Can we also loop over dicts?}
Sure! But we need to indicate how exactly:

\begin{lstlisting}
mydict = {"A":100, "B": 60, "C": 30}

for k in mydict:   # or mydict.keys()
    print(k)

for v in mydict.values():
    print(v)

for k,v in mydict.items():
    print(f"{k} has the value {v}")
\end{lstlisting}

\end{frame}




\begin{frame}[fragile]{Can we also loop over dicts?}
The result:

\begin{lstlisting}
A
B
C

100
60
30

A has the value 100
B has the value 60
C has the value 30
\end{lstlisting}

\end{frame}



\begin{frame}[fragile]{if statements}
	\begin{block}{Structure}
		Only execute block if condition is met
	\end{block}
	\begin{lstlisting}
x = 5
if x <10:
   print(f"{x} is smaller than 10")
elif x > 20:
   print(f"{x} is greater than 20")
else:
   print("No previous condition is met, therefore 10<={x}<=20")
\end{lstlisting}

\end{frame}


\question{Can you see how such an if statement could be particularly useful when nested in a for loop?}



\begin{frame}[fragile]{try/except}
\begin{block}{Structure}
If executed block fails, run another block instead
\end{block}
\begin{lstlisting}
x = "5"
try: 
   myint = int(x)
except:
   myint = 0
\end{lstlisting}

\pause 
\small{Again, more useful when executed repeatedly (in a loop or function):}
\begin{lstlisting}
mylist = ["5", 3, "whatever", 2.2]
myresults = []
for x in mylist:
    try: 
        myresults.append(int(x))
    except:
        myresults.append(None)
print(myresults)
\end{lstlisting}
\end{frame}





\section{Bonus: Python goodies}

\begin{frame}[fragile]{List comprehensions}
\begin{block}{Structure}
A for loop that \texttt{.append()}s to an empty list can be replaced by a one-liner:
\end{block}
\begin{lstlisting}
mynumbers = [2,1,6,5]
mysquarednumbers = []
for x in mynumbers:
    mysquarednumbers.append(x**2))
\end{lstlisting}
is equivalent to:
\begin{lstlisting}
mynumbers = [2,1,6,5]
mysquarednumbers = [x**2 for x in mynumbers]
\end{lstlisting}

\pause 
Optionally, we can have a condition:
\begin{lstlisting}
mynumbers = [2,1,6,5]
mysquarednumbers = [x**2 for x in mynumbers if x>3]
\end{lstlisting}

\end{frame}




\begin{frame}[fragile]{List comprehensions}
	\begin{alertblock}{A very pythonic construct}
		\begin{itemize}
			\item Every for loop can also be written as a for loop that appends to a new list to collect the results. 
			\item For very complex operations (e.g., nested for loops), it can be easier to write out the full loops. 
			\item But mostly, list comprehensions are really great! (and much more concise!)
		\end{itemize}
	\end{alertblock}
\textbf{$\Rightarrow$ You really should learn this!}
	\end{frame}



\begin{frame}[fragile]{Generators}
\begin{block}{Structure}
	A lazy for loop (or function) that only generates its next element when it is needed:
\end{block}
\footnotesize
You can create a generator just like a list comprehension (but with \texttt{()} instead of \texttt{[]}):
\begin{lstlisting}
mynumbers = [2,1,6,5]
squaregen = (x**2 for x in mynumbers)   # these are NOT calculated yet
for e in squaregen:
    print(e)             # only here, we are calculating the NEXT item
\end{lstlisting}
\pause
Or like a function (but with \texttt{yield} instead of \texttt{return}):
\begin{lstlisting}
def squaregen(listofnumbers):
    for x in listofnumbers:
        yield(x**2)
mygen = squaregen(mynumbers)
for e in mygen:
    print(e)
\end{lstlisting}

\end{frame}


\begin{frame}[fragile]{Generators}
	\begin{alertblock}{A very memory and time efficient construct}
		\begin{itemize}
			\item Every function that \emph{returns} a list can also be written as a generator that \emph{yields} the elements of the list
			\item Especially useful if
			\begin{itemize}
				\item it takes a long time to calculate the list
				\item the list is very large and uses a lot of memory (hi big data!)
				\item the elements in the list are fetched from a slow source (a file, a network connection)
				\item you don't know whether you actually will need all elements
			\end{itemize}
		\end{itemize}

		\end{alertblock}
		\textbf{$\Rightarrow$ You probably don't need this right now, but (a) it will come in very handy once you deal with web scraping or very large collections, and (b) you may come across generators in some examples}
		\end{frame}



\begin{frame}[standout]
	Make sure you understood all of today's concepts.
	
	Also read Chapter 4 and ask questions on Friday if needed.
	
	 I prepared exercises to work on \emph{during} the Friday meeting (alone or in teams):
	 \large{\url{https://github.com/damian0604/bdaca/blob/master/12ec/week02/exercises/exercises.md}}
\end{frame}


%\begin{frame}[plain]
%    \printbibliography
%\end{frame}

\end{document}

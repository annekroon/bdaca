
The following schedule gives an overview of the topics covered each week, the obligatory literature that has to be studied each week, and other tasks the students have to complete in preparation of the class.
In particular, the schedule shows which chapter of \cite{cssbook} will be dealt with. Note that some basic chapters that explain how to install the software we are going to use have to be read before the course starts.

Next to the obligatory literature, the following books provide the interested student with more and deeper information. They are intended for the advanced reader and might be useful for final individual projects, but are by no means required literature. Bear in mind, though, that you may encounter slightly outdated examples (e.g., Python 2, now-defunct APIs etc.).

\begin{itemize}
%\item \citealp{Russel2013}. Gives a lot of examples about how to analyze a variety of online data, including Facebook and Twitter, but going much beyond that.
%\item \citealp{Bird2009}. This is the official documentation of the NLTK package that we are using. A newer version of the book can be read for free at \url{http://nltk.org}
\item \citealp{McKinney2012}: A lot of examples for data analysis in Python. A PDF of the book can be downloaded for free on \url{http://it-ebooks.info/book/1041/}.
\item \citealp{VanderPlas2016}: A more recent book on numpy, pandas, scikit-learn and more. It can also be read online for free on \url{https://jakevdp.github.io/PythonDataScienceHandbook/}, and the contents are avaibale as Jupyter Notebooks as well \url{https://github.com/jakevdp/PythonDataScienceHandbook}.
\item The pandas cookbook by Julia Evans, a collection of notebooks on github: \url{https://github.com/jvns/pandas-cookbook}.
\item \citealp{Hovy2020}: A thin book on bottom-up text analysis in Python with both a bit more math background and ready-to-use Python code implementations.
\item \citealp{Salganik2017}: Not a book on Python, but on research methods in the digital age. Very readable, and a lots of inspiration and background about techniques covered in our course.
\end{itemize}

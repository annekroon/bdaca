\begin{corona}\noindent \textbf{Changes due to corona-related online reaching.} You may have heard from students who took this course in the last years about the general setup: one lecture and one lab session per week. During the lab session, students were working through the chapters in the (old) book,  additional ressources, or their own analyses. I was walking around, helping on a 1:1 basis, and when I realized that multiple students had the same problem, I was explaining it plenarily. Over the last years, student evaluations have consistently shown that the format was very much appreciated. At the same time, it is very hard to replicate this format online.
	This is how we will do it:

	\begin{itemize}
		\item You will need to thoroughly read through the materials \textbf{before} each meeting.
		\item Until Wednesday morning, you can submit questions that I will spend some time answering on during the lab sessions.
		\item During the lab sessions, we will use breakout rooms in which you will can discuss your work and problems with your classmates. The goal here is to develop problem-solving strategies together.
		\item We will use Zoom's ``Remote Support''/``Remote Control'' feature, that allows people to take over each other's keyboard, mouse, and screen (of course you need to approve), so that you can code together.
		\item A second teacher, Vladislav Petkevich, will assist during the course. Both Vladislav and I will ``walk around'' the breakout rooms and help out.
		\item Vlasislav will offer online office hours to deal with specific technical problems.
	\end{itemize}
\end{corona}

\section*{Before the course starts: Prepare your computer.}
\textsc{\ding{52} Chapter 1: Introduction}\\
Make sure that you have a working Python environment installed on your computer. You cannot start the course if you have not done so.

\begin{corona}
	Each week:
	\begin{itemize}
		\item Read the book chapter and/or literature \emph{before} the Monday session
		\item Submit questions for the Thursday meetings no later than Wednesday morning
		\item Work on writing code during Friday sessions; ask questions in breakout rooms
	\end{itemize}
\end{corona}

\section*{Schedule}

\section*{Week 1: What is Computational Social Science, and why Python?}
\subsection*{Monday, 29--3. Lecture.}
We discuss what Big Data and Computational (Social|Communication) Science are. We talk about challenges and opportunities as well as the implications for the social sciences in general and communication science in particular. We also pay attention to the tools used in CSS, in particular to the use of Python.

Mandatory readings (in advance):  \cite{boyd2012}, \cite{Kitchin2014}, \cite{Hilbert2019}.

Additionally, the journal \textit{Commmunication Methods and Measures} had a special issue (volume 12, issue 2--3) about Computational Communication Science. Read at least the editorial \citep{VanAtteveldt2018a}, but preferably, also some of the articles (you can also do that later in the course).


\subsection*{Thursday, 4--3. Lab session.}
\textsc{\ding{52} Chapter 2: Fun with data}\\

During the lab session, we will run our first code. We will showcase some possibilities, and leave the technical background for next week.

\section*{Week 2: Getting started with Python  }

\subsection*{Monday, 5--4: Eastern Monday}

\subsection*{Thursday, 8--4. Lecture.}
\textsc{\ding{52} Chapter 3: Programming concepts for data analysis}\\
You will get a very gentle introduction to computer programming. During the lecture, you are encouraged to follow the examples on your own laptop.


\section*{Week 3: Getting started with Python (continued)  and Data formats}

We talk about file formats such as \texttt{csv} and \texttt{json}; about encodings; about reading these formats into basic Python structures such as dictionaries and lists as opposed to reading them into dataframes; and about retrieving such data from local files, as parts of packages, and via an API.

\subsection*{Monday, 12--4. Lab session}
\textsc{\ding{52} Chapter 4: How to write code}\\
We will do our first real steps in Python and do some exercises to get the feeling.\\ 

\subsection*{Thursday, 15--4. Lecture plus lab session}
\textsc{\ding{52} Chapter 5: From file to dataframe and back}\\
\textsc{\ding{52} Chapter 12.1: Using web APIs: from open resources to Twitter}\\
A conceptual overview of different file formats and data sources, and some practical guidance on how to handle such data in basic Python and in Pandas. We will practice with writing a script to collect and handle some JSON data.


\section*{Week 4: Data wrangling, simple statistics and visualizations}

Of course, you don't need Python to do statistics. Whether it's R, Stata, or SPSS -- you probably already have a tool that you are comfortable with. But you also do not want to switch to a different environment just for getting a correlation. And you definitly don't want to do advanced data wrangling in SPSS\ldots
This week, we will discuss different ways of organizing your data (e.g., long vs wide formats) as well as how to do conventional statistical tests and simple plots in Python.

\subsection*{Monday, 19--4. Short lecture plus lab session.}
\textsc{\ding{52} Chapter 6: Data wrangling}\\
We will learn how to do data wrangling with pandas: converting between wide and long formats (melting and pivoting), aggregating data, joining datasets, and so on.

\subsection*{Thursday, 22--4.  Short lecture plus lab session.}
\textsc{\ding{52} Chapter 7.1. Simple exploratory data analysis}\\
\textsc{\ding{52} Chapter 7.2. Visualizing data}\\

\section*{Week 5: Working with text}

In this week, we will dive into how to deal with textual data. How is text represented, how can we clean it, and how can we extract useful information from it?

\subsection*{Monday, 26--4: Kingsday}

\subsection*{Thursday, 29--4. Lecture plus lab session}
\textsc{\ding{52} Chapter 9: Processing text}\\
We discuss basic string operations and regular expressions. You will write a script to conduct a top-down automated content analysis, in which you check for the occurrence of predefined patterns or strings, and extract data from text based on regular expressions.

\subsection*{Take-home exam}
In week 5, the first midterm take-home exam is distributed after the Thursday meeting. The answer sheets and all files have to be handed in no later than Monday evening (3--5, 23.59).

\section*{Week 6: No Teaching}
\subsection*{Monday, 3--5: Teaching-free week UvA}
\subsection*{Thursday, 6--5: Teaching-free week UvA}

\section*{Week 7: (Clean) representations of text}

Text as written by humans usually is pretty messy. You can use some of the techniques you learned last week to clean it up (e.g., to remove punctuation), but in this week, we will dive a bit deeper into ways to represent text in a clean(er) way. We will introduce the Bag-of-Words (BOW) representation and show multiple ways of transforming text into matrices.

\subsection*{Monday, 10--5. Lecture.}
\textsc{\ding{52} Chapter 10: Text as data}\\
This lecture will introduce you to techniques and concepts like stemming, stopword removal, n-grams, word counts and word co-occurrances, and regular expressions. We will do some exercises during the lecture.

Preparation: Mandatory reading: \cite{Boumans2016}.

\subsection*{Thursday, 13--5. Lab session.}
You will combine the techniques discussed on Monday and write a full automated content analysis script using a top-down dictionary or regular-expression approach.

\section*{Week 8: Machine learning}
During the final week, we will discuss the basics of machine learning. You will be introduced to \citep{scikit-learn}, one of the most well-known machine learning libraries. We do not have the time to discuss machine learning techniques in depth. Rather, a practical and hands-on introduction is provided. 

\subsection*{Monday, 17--5. Lecture}
\textsc{\ding{52} Chapter 8: Statistical Modeling and Supervised Machine Learning}\\
\textsc{\ding{54} (you can skip 8.4 Deep Learning for now)}\\

We will discuss the basics of supervised machine learning, and how its performance can be evaluated. 

\subsection*{Thursday, 20--5. Lab session.}
We exercise with supervised machine learning as a technique for automated content analysis. Possibility to ask last (!) questions regarding the final project.


\subsection*{Final project}
Deadline for handing in: Friday, 28--5, 23.59.




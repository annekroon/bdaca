\documentclass{beamer}
%\documentclass[handout]{beamer}

%\setbeamertemplate{background canvas}[vertical shading][bottom=white,top=structure.fg!25]
% or whatever

\usetheme[compress]{Amsterdam}
%\setbeamertemplate{headline}{}
%\setbeamertemplate{footline}{}
%\setbeamersize{text margin left=0.5cm}
  
\usepackage[english]{babel}
\usepackage{listings}
\usepackage{geometry}
\usepackage{hyperref}

\usepackage{color}

\usepackage[utf8]{inputenc}
\usepackage[T1]{fontenc}
\usepackage{lmodern}

\lstset{
basicstyle=\scriptsize\ttfamily,
columns=flexible,
breaklines=true,
numbers=left,
%stepsize=1,
numberstyle=\tiny,
backgroundcolor=\color[rgb]{0.85,0.90,1}
}


\begin{document}

<<<<<<< HEAD
\title[Big Data and Automated Content Analysis]{\textbf{Big Data and Automated Content Analysis} \\ Week 2 -- Monday\\ »Getting started with Python«}
\author[Anne Kroon]{Anne Kroon \\ ~ \\ \footnotesize{a.c.kroon@uva.nl \\@annekroon} \\}
\date{April 8, 2019}
=======
\title[Big Data and Automated Content Analysis]{\textbf{Big Data and Automated Content Analysis} \\ Week 2 -- Monday \\ »Getting started with Python«}
\author[Damian Trilling]{Damian Trilling \\ ~ \\ \footnotesize{d.c.trilling@uva.nl \\@damian0604} \\ \url{www.damiantrilling.net}}
\date{6 April 2020}
>>>>>>> 314096216d9b77583e382c099caf965fd8700d10
\institute[UvA]{Afdeling Communicatiewetenschap \\Universiteit van Amsterdam}


%\begin{frame}[plain]
%\huge{\textbf{You are encouraged to start up a Python environment (like Spyder or Jupyter Notebook)}. If you do so, you can try out the examples while listening. If you prefer to listen only, that's fine as well.}
%\end{frame}


\begin{frame}{}
\titlepage
\end{frame}

\begin{frame}{Today}
\tableofcontents
\end{frame}


\section[Basics]{The very, very, basics of programming with Python}
\begin{frame}[plain]
\textbf{The very, very, basics of programming}\\
\vspace{1cm}
See also Chapter 4.
\end{frame}
\subsection{Datatypes}


\begin{frame}{Python lingo}
\begin{block}{Basic datatypes (variables)}
\begin{description}
\item[{\color{red}int}] \texttt{32}
\item[{\color{red}float}] \texttt{1.75}
\item[{\color{red}bool}] \texttt{True}, \texttt{False}
\item[{\color{red}string}] \texttt{"Jessica"}
\onslide<2->{\scriptsize \item[({\color{red}variable name}] \texttt{firstname})}
\end{description}
\end{block}
\onslide<2->{\textbf{"firstname" and firstname is not the same.\\}}
\onslide<3->{\textbf{"5" and 5 is not the same.}\\ But you can transform it: {\tt{int("5")}} will return 5.}\\
\onslide<3->{\textbf{You cannot calculate \texttt{3 * "5"}} {\tiny{(In fact, you can. It's \tt{"555"})}}.\\
But you can calculate {\tt{3 * int("5")}}}
\end{frame}


\begin{frame}{Python lingo}
\begin{block}{More advanced datatypes}
\begin{description}
\item[{\color{red}list}]<2-> \texttt{firstnames = $[$'Damian','Lori','Bjoern'$]$ \\ lastnames = $[$'Trilling','Meester','Burscher'$]$}
\item[{\color{red}list}]<3->\texttt{ages = $[$18,22,45,23$]$}
\item[{\color{red}dict}]<4-> \texttt{familynames= \{'Bjoern': 'Burscher', 'Damian': 'Trilling', 'Lori': 'Meester'\} }
\item[{\color{red}dict}]<4-> \texttt{\{'Bjoern': 26, 'Damian': 31, 'Lori': 25\} }

\end{description}
\pause
Note that the elements of a list, the keys of a dict, and the values of a dict can have any datatype! (Better to be consistent, though!)
\end{block}
\end{frame}


\subsection{Functions and methods}
\begin{frame}{Python lingo}
\begin{block}{Functions}
\begin{description}
\item[{\color{red}functions}]<2-> Take an input and return something else \\ {\tt{int(32.43})} returns the integer 32. \texttt{len("Hello")} returns the integer 5.\\ 
\item[{\color{red}methods}]<3-> are similar to functions, but directly associated with an object. {\tt{"SCREAM".lower()}} returns the string "scream"
\end{description}
\end{block}
\onslide<4->{Both functions and methods end with \texttt{()}. Between the \texttt{()}, \emph{arguments} can (sometimes have to) be supplied.}
\end{frame}


\begin{frame}[fragile]{Writing own functions}
You can write an own function:
\begin{lstlisting}
def addone(x):
    y = x + 1
    return y
\end{lstlisting}

Functions take some input (``argument'') (in this example, we called it \texttt{x}) and \emph{return} some result.
	
Thus, running
\begin{lstlisting}	
addone(5)
\end{lstlisting}
returns \tt{6}.
\end{frame}



\subsection{Modifying lists and dictionaries}

\begin{frame}[plain]
Modifying lists and dictionaries
\end{frame}{}
	
\begin{frame}[fragile]{Modifying lists}
\begin{block}{Appending to a list}
\begin{lstlisting}
mijnlijst = ["element 1", "element 2"]
anotherone = "element 3"   # note that this is a string, not a list!
mijnlijst.append(anotherone)
print(mijnlijst)
\end{lstlisting}
gives you:
\begin{lstlisting}
["element 1", "element 2", "element 3"]
\end{lstlisting}
\end{block}
\end{frame}



\begin{frame}[fragile]{Modifying lists}
\begin{block}{Merging two lists (= extending)}
\begin{lstlisting}
mijnlijst = ["element 1", "element 2"]
anotherone = ["element 3", "element 4"]
mijnlist.extend(anotherone)
print(mijnlijst)
\end{lstlisting}
gives you:
\begin{lstlisting}
["element 1", "element 2", "element 3", "element 4]
\end{lstlisting}
\end{block}
\end{frame}




\begin{frame}[fragile]{Modifying dicts}
\begin{block}{Adding a key to a dict (or changing the value of an existing key)}
\begin{lstlisting}
mydict = {"whatever": 42, "something": 11}
mydict["somethingelse"] = 76
print(mydict)
\end{lstlisting}
gives you:
\begin{lstlisting}
{'whatever': 42, 'somethingelse': 76, 'something': 11}
\end{lstlisting}
If a key already exists, its value is simply replaced.
\end{block}
\end{frame}




\subsection[Indention]{Indention: The Python way of structuring your program}
\begin{frame}[plain]
Indention: The Python way of structuring your program
\end{frame}


\begin{frame}[fragile]{Indention}
\begin{block}{Structure}
The program is structured by TABs or SPACEs
\end{block}

\end{frame}



{\setbeamercolor{background canvas}{bg=black}
\begin{frame}[plain]
\makebox[\linewidth]{
\includegraphics[width=\paperwidth,height=\paperheight,keepaspectratio]{../../pictures/tabsvsspaces}
}
\end{frame}
}

\begin{frame}[fragile]{Indention}
\begin{block}{Structure}
The program is structured by TABs or SPACEs
\end{block}
\begin{lstlisting}
firstnames=['Anne','Lori','Bjoern']
age={'Bjoern': 27, 'Anne': 33, 'Lori': 26}
print ("The names and ages of these people:")
for naam in firstnames:
    print (naam,age[naam])
\end{lstlisting}
\onslide<2->{\textbf{Don't mix up TABs and spaces! Both are valid, but you have to be consequent!!! Best: always use 4 spaces!}}
\end{frame}





\begin{frame}[fragile]{Indention}
\begin{block}{Structure}
The program is structured by TABs or SPACEs
\end{block}
\begin{lstlisting}
print ("The names and ages of all these people:")
for naam in firstnames:
    print (naam,age[naam])
    if naam=="Anne":
        print ("She teaches this course")
    elif naam=="Lori":
        print ("She is a former assistant")
    elif naam=="Bjoern":
        print ("He helped teaching this course in the past")
    else:
        print ("No idea who this is")
\end{lstlisting}
\end{frame}


\begin{frame}{Indention}
The line \emph{before} an indented block starts with a \emph{statement} indicating what should be done with the block and ends with a \texttt{:}

\begin{block}{Indention of the block indicates that}<2->
\begin{itemize}
\item<3-> it is to be executed repeatedly (\texttt{for} statement) – e.g., for each element from a list
\item<4-> it is only to be executed under specific conditions (\texttt{if}, \texttt{elif}, and \texttt{else} statements)
\item<5-> an alternative block should be executed if an error occurs (\texttt{try} and \texttt{except} statements)
\item<6-> a file is opened, but should be closed again after the block has been executed (\texttt{with} statement)
\end{itemize}
\end{block}
\end{frame}


\section{Exercise}
\subsection*{Exercise}
\begin{frame}
We'll now together do some simple exercises \ldots
\end{frame}



\begin{frame}{Exercises}
\begin{block}{1. Warming up}
\begin{itemize}
	\item Create a list, loop over the list, and do something with each value (you're free to choose). 
\end{itemize}
\end{block}
\begin{block}{2. Did you pass?}
	\begin{itemize}
	\item Think of a way to determine for a list of  grades whether they are a pass (>5.5) or fail.
	\item Can you make that program robust enough to handle invalid input (e.g., a grade as 'ewghjieh')?
	\item How does your program deal with impossible grades (e.g., 12 or -3)?
	\item \ldots
\end{itemize}
\end{block}
\end{frame}

\section{Next meetings}
\begin{frame}
Next meetings
\end{frame}


\begin{frame}{Thursday}
<<<<<<< HEAD
	We will work together on ``Describing an existing structured dataset'' (Appendix A).\\
	\textbf{Preparation: Make sure you understood all of today's concepts!}
=======
	Do and ask questions in advance about the exercise ``Describing an existing structured dataset'' (Appendix A).
>>>>>>> 314096216d9b77583e382c099caf965fd8700d10
\end{frame}


%\begin{frame}{Week 3: Data harvesting and storage}
%\begin{block}{Monday, 13--4}
%A conceptual overview of APIs, scrapers, crawlers, RSS-feeds, databases, and different file formats
%\end{block}
%
%\begin{block}{Wednesday, 15--4}
%Writing some first data collection scripts
%\end{block}
%
%\begin{block}{Preparation}
%\begin{itemize}
%\item Conceptual level: Read the article by Morstatter, Pfeffer, Liu, and Carley (2013) about the limitations of the Twitter API. 
%\item Technical level: \textbf{Make sure you are comfortable with the techniques we've covered so far.} Play around. Give yourself some tasks and solve them. Google. 
%\end{itemize}
%\end{block}
%
%
%\end{frame}

\end{document}


